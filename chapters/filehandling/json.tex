% !TeX root = ../../pythonTutorial.tex
\section{JSON}
\label{filehandling:section:json}

JavaScript Object Notation (JSON) ist ein Format f�r den Austausch von Daten, welches unabh�ngig von der Programmiersprache ist. Aufgrund von Konventionen,
die dieses Format mit Programmiersprachen aus der C-Familie, wie C, C++, Java oder Python, teilt, liefert es eine Programmierern bekannte Textstruktur.

In Python 3 ist nativ das json-Package enthalten, welches das Arbeiten mit
dem JSON-Format erm�glicht. Mit Hilfe des folgenden Codes binden wir das Package in das Projekt ein. 

%Code
% import json

Ein gegebener JSON-String wird �ber die \lstinline$loads()$-Methode in ein in Python existierendes, entsprechendes Objekt geparst.
In diesem Fall wird ein Dictionary angelegt.

JSON zu Python
%Code
% student = '{"name":"Student", "age":"24", "height":"1.8"}'
% studentDict = json.loads(student)
% print(studentDict["age"])

F�r das Umwandeln eines Python-Objekts in einen JSON-String, verwenden wir die 
\lstinline$dumps()$-Methode.

Python zu JSON

%Code

Konvertieren wir Python- zu JSON-Objekte, werden diese im JSON-�quivalent (JavaScript) angelegt.

%Code

Wenn wir einen Dictionary mit mehreren Schl�ssel-Objekt-Paaren anlegen,
werden wir bei der Ausgabe des JSON-Objekts feststellen, dass diese auf eine Zeile beschr�nkt ist. 

%Code

Zur Formatierung unserer Ausgabe verwenden wir die \lstinline$dumos()$-Methode.
Mit Hilfe des \lstinline$indent$-Parameters k�nnen festlegen, ob und wie weit die Textstruktur einger�ckt werden soll. 
Der \lstinline$separators$-Parameter legt die Trennzeichen fest und mit \lstinline$sort_keys=True$ wird die Ausgabe der Schl�ssel lexikographisch sortiert.
  
%Code



