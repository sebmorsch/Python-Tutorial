
\section{Grundlagen}
Die Programmiersprache Python wurde Anfang der 1990er Jahre von Guido van Rossum entwickelt.
Der Name der Sprache beruht auf der Komikergruppe  Monty Python.
Hierzu lassen sich auch Zahlreiche Anspielung in der Dokumentation von Python finden.
Python wurde mit dem Ziel größter Einfachheit sowie Übersichtlichkeit entworfen.
Dies wird nicht zuletzt durch die große übersichtlichkeit in der Standardbibliothek versucht zu erreichen sondern auch durch die Modulare Erweiterbarkeit.

%Python ist eine Scriptsprache die 
%scriptsprache
%anwendungszweck

\subsection{Installation des Interpreters}
Python kann auf der Webseite https://www.python.org bezogen werden.
Im folgenden wird die Python in der Version 3 behandelt.





\subsection{Syntax}
Folgende syntaktische Besonderheiten Bringt Python mit:
\subsubsection{Leerzeichen und Einrückung}
Um in Python Blöcke auszuzeichnen können nicht wie in Java und C++ geschweifte Klammern genutzt werden. 
In Python ist hierfür entweder der Tabulator oder 4 aufeinander folgende Leerzeichen vorgesehen.
Somit kann bei zum Beispiel einer IF-Abfrage der nachfolgende Block leicht falsch zugeordnet werden wenn in den nachfolgenden Zeilen die Einrückung übersehen wird.
\subsubsection{Kommentare}
Innerhalb Python wird zwischen Zeilen und Blockkommentaren unterschieden.
Zeilenweise Kommentare werden über das Rautensymbol (\# ) eingeleitet.
Blockkommentare über drei aufeinander folgenden Anführungzeichen ('''''').
Im folgenden jeweils ein Beispiel für Zeilenweise Kommentare sowie Blockkommentare.
\lstinputlisting{chapters/sections/listings/comment.py}
\subsubsection{Typsicherheit}
Anders als bei Java und C++ ist Python eine nur schwach typisierte Sprache.
Somit ist bei der Initialisierung keine Typangabe erforderlich. 
Der Datentyp wird beim Initialisieren dynamisch ermittelt und automatisch zugewiesen.
\subsubsection{prozedurale Programmierung}
\subsection{Interpreter}
%übersetz das ihm übergebene Script
%kann als Umgebung genutzt werden


\subsection{Beispiel \glqq Hallo World!\grqq}
Hier ein einfaches \glqq Hallo World!\grqq -Beispiel.

\lstinputlisting{chapters/sections/listings/helloworld.py}
