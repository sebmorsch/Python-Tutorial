% !TeX root = ../../pythonTutorial.tex
\section{Threads}

\label{threads}

In Python werden zwei APIs zur Verwendung von Threads angeboten, die Low-Level API aus dem \_thread-Modul und die Higher-Level API aud dem threading-Modul. Es wird sich an dieser Stelle auf das threading-Modul beschr�nkt, da es Intern auf dem \_thread-Modul basiert und eine Schnittstelle anbietet, welche das Programmieren von Multithreaded Applikationen erleichert. Diese Schnittstelle ist an der Thread-Schnittstelle aus Java angelehnt und sollte daher f�r Java-Entwickler leicht zu verwenden sein. Allerdings gibt es einige Unterschiede zwischen dem Python-Modul und der Java-Implementierung. So sind Bedingungsvariablen und Locks seperate Objekte in Python und es ist auch nur eine Teilmenge des Verhaltens eine Java-Threads in Python verf�gbar. Ein Python-Thread kennt keine Priorit�ten und Thread-Gruppen und er kann nicht zerst�rt, gestopped, angehalten, fortgesetzt oder unterbrochen werden. Soweit vorhanden sind die statischen Methoden aus der Java-Thread-Klasse auf Modul-Ebene in Python implementiert.