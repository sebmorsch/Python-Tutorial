% !TeX root = ../pythonTutorial.tex

\section{Elementare Datentypen}

�hnlich wie zu Java und C oder C++, gibt es auch in Python Variablen. Allerdings gibt es dabei immense Unterschiede zu den anderen Programmiersprachen, weshalb sich ein genauerer Blick auf die einzelnen Datentypen in jedem Fall lohnt.
In den uns bekannten Sprachen muss eine Variable einem bestimmten Datentyp zugeordnet (deklariert) werden. Der Datentyp kann darauf folgend zur Laufzeit nicht wieder ge�ndert werden, der Wert innerhalb des Datentyps allerdings schon. Ist eine Variable einmal zum Beispiel dem Datentyp Integer zugeordnet, so l�sst sich der Variable anschlie�end nicht mehr in einen String-Wert umwandeln. In Python hingegen, ist dies ohne weiteres m�glich. Hier wird n�mlich g�nzlich auf eine explizite Typdeklaration verzichtet. Zeigt eine Variable beispielsweise auf eine ganze Zahl, so wird diese als ein Objekt vom Typ Integer interpretiert. Allerdings kann man sie im n�chsten Schritt einfach auf ein String-Objekt zeigen lassen. 
Soviel zum allgemeinen Unterschied zu den anderen Programmiersprachen. Betrachten wir nun die Datentypen etwas genauer.
