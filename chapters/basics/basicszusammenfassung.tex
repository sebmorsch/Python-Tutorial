% !TeX root = ../../pythonTutorial.tex

\section{Grundlagen Zusammenfassung} 
\label{grundlagenzs:sec:Zusammenfassung}
Abschlie�end soll das Kapitel Grundlagen noch einmal f�r den Leser zusammengefasst werden.
Zu Beginn des Kapitels hat der Leser seine ersten Schritte mit der Sprache Python gemacht. 
Es wurde grundlegend vorgestellt, wobei es sich bei Python �berhaupt handelt und wie die Programmiersprache installiert wird. 
Der Python Interpreter vorgestellt, einer einfachen Konsolenanwendung zur Ausf�hrung von Python Code. 
Weiterhin wurden Kommentare und die spezielle Blockstruktur von Python eingef�hrt. 
Um das erste funktionierende Programm mit dem bis dahin erlangten Wissen erstellen, wurde ein klassisches Hello-World! Programm geschrieben. 
Hierbei wurde auch die Einfachheit von Python im Zusammenhang mit fehlenden Klassenkonstrukten wie bei Java oder C++ erl�utert.
Auch wenn sich einfache Konzepte und Funktionen noch leicht in einem Texteditor oder der Standard IDE umsetzen lassen, lohnt es sich, eine IDE mit zus�tzlichem Funktionsumfang zu nutzen.
Hierzu wurden zuerst w�nschenswerte Funktionen, deren Nutzen erl�utert und anschlie�end passende IDE's vorgestellt. 
Die Autoren empfehlen ausdr�cklich die Verwendung einer IDE bei der Bearbeitung der Aufgaben in den folgenden Kapiteln.
Dabei sollte sich der Leser eine IDE aussuchen, welche ihm zusagt und ihn optimal beim Entwickeln von Python Code unterst�tzt. 
Als Grundlage f�r die weitere Arbeit mit Python wurden die verschiedenen einfachen Datentypen betrachtet.
Dabei wurden neben den Zahlenwerten, Boolean- und Stringvariablen auch Aufz�hlungen mit der Hilfe von ENUMs gezeigt.
Dem Leser wurde ebenfalls das neutrale Element vorgestellt, welches an verschiedenen Stellen zum Einsatz gebracht werden kann.
Im Zusammenhang mit einfachen Datentypen wurde auch eine Besonderheit von Python erl�utert, die M�glichkeit der Referenzierung eines Objektes durch eine Variable, ohne dabei einen festen Typ zuzuordnen.
Ein wichtiger Bestandteil einer Programmiersprache sind die zugeh�rigen Kontrollstrukturen.
In diesem Zusammenhang wurde der Umgang mit Abfragen gezeigt.
Zum einen wurde dabei betrachtet wie man if-else Anweisungen einsetzt und wie alternativ dazu Conditional Expressions verwendet werden k�nnen.
Im weiteren wurde die Benutzung von Schleifen in der Python-Programmierung gezeigt.
Zum Abschluss des Themas Kontrollstrukturen wurde kurz eine allgemeine �bersicht zu den g�ngigen logischen Operatoren und deren Nutzen geliefert.
Im abschlie�enden Kapitel der Grundlagen wurden Collections zur Aufbewahrung von Daten und deren Verwendung in Python beschrieben. Collections k�nnen je nach Datenstruktur unterschiedliche Eigenschaften aufweisen. Dem Leser wurden die verschiedenen Eigenschaften, zugeh�rige Funktionen und die Anwendungsm�glichkeiten vorgestellt. Zu den gezeigten Datentypen geh�ren List, Tuple, Set und Dictionary. 
Durch das Wissen, welches der Leser im Kapitel Grundlagen erlangt hat, ist dieser nun bereit einfache Konzepte von Python anzuwenden und zu verstehen. Es wurden eine Vielzahl an �bungen bereitgestellt, um den Lernprozess zu unterst�tzen. Da es sich hierbei um essentielle Grundlagen von Python handelt, werden diese f�r alle weiteren Kapitel vorausgesetzt.
Es wird deshalb empfohlen, erst mit dem Tutorial fortzufahren, wenn dieses Kapitel vollst�ndig abgeschlossen ist.
