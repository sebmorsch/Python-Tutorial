% !TeX root = ../pythonTutorial.tex
\chapter{Weiterf�hrende Themen}
\section{Maschinelles Lernen in Python}
Python bietet im Bereich des maschinellen Lernens unterschiedliche M�glichkeiten an. Eine weit verbreitete M�glichkeit ist das Einbinden von Module, die bereits gewissen Funktionalit�ten von Haus auf anbieten. Eine zweite M�glichkeit ist die Integration der Programmiersprache R. Beide M�glichkeiten werden im Folgenden weiter beschrieben.

\subsection{Datenanlyse, Visualisierung und Pr�sentation}
Die folgenden Module erweitern die Standardfunktionalit�t und geben so die M�glichkeit numerische L�sungen zu mathematischen Problemen zu erzeugen.

Zwei bekannte Module sind \textit{numpy} und \textit{scipy}, mit denen beispielsweise Gleichungen und Optimierungsprobleme gel�st werden k�nnen. Au�erdem lassen sich Integrale berechnen, statistische Berechnungen durchf�hren und auch simulieren. Als das wird f�r maschinelles Lernen ben�tigt. Da die Berechnungen mit Routinen nah an der Hardware durchgef�hrt werden, lassen sich bei entsprechender Programmierung effiziente Programme schreiben. Ergebnisse solcher Berechnungen k�nnen mithilfe des Moduls \textit{matplotlib} visualisiert werden. ~\cite{Python3}

%https://www.david-benjamin-hentschel.de/maschinelles-lernen-mit-python/

%https://data-science-blog.com/blog/2015/05/24/top-10-der-python-bibliotheken-fur-data-science/

Die Module m�ssen meist erst installiert werden, bevor sie genutzt werden k�nnen.

\textbf{numpy}\\
In diesem Modul wird ein flexibler Datentyp f�r mehrdimensionale Arrays zur Verf�gung gestellt. Dies erm�glicht eine effiziente Durchf�hrung von komplexen Rechnungen.
~\cite{numpyreference}

\textbf{scipy}\\
Erg�nzend bzw. aufbauen auf numpy werden durch scipy viele mathematische Operationen bereit gestellt. 

Das Modul scipy ist sehr m�chtig und daher nochmal in Untermodule aufgeteilt. Innerhalb der Untermodule werden bestimmte Funktionalit�ten gruppiert. Die genauen Untermodule k�nnen aus der Onlinedokumentation ~\cite{scipyreference}

\textbf{pandas}\\
andas is a Python package providing fast, flexible, and expressive data structures designed to make working with relational or labeled data both easy and intuitive. ~\cite{pandas}

\textbf{matplotlib}\\
Mit dem Modul matplotlib k�nnen Daten in einem Diagramm dargestellt werden. Hiermit kann ein erstes Verst�ndnis der Daten oder Ergebnisse erreicht werden. Es werden unter anderem Liniendiagramme, Histogramme, Balkendiagramme aber auch Heatmaps unterst�tzt. Hier k�nnen sowohl Achsen, Farben und auch Beschriftungen nach Bedarf angepasst werden.

Der Link zur Modulseite findet sich im Literaturverzeichnis unter: ~\cite{matplotlib}




\subsubsection{Machine Learning}
Um die Integration der oben genannte zu Verdeutlichen wird im folgenden ein Beispiel dargestellt.

\begin{lstlisting}
import numpy as np
import scipy as sc
import matplotlib as mpl
\end{lstlisting}



\textbf{R-Integration}\\
Der zweite Ansatz ist die Integration von R-Komponenten. 



\begin{lstlisting}
import rpy2.robjects as R
\end{lstlisting}




\subsubsection{Erweiterungsm�glichkeiten}

SymPy is a Python library for symbolic mathematics. 





%Datenanlyse, Visualisierung und Pr�sentation
%    Numpy
%    Matplotlib
%    Pandas 
%Machine Learning
%    Machine Learning: Terminologie
%    Einf�hrung in Scikit
%    k-nearest Neighbor Classifier
%    Einf�hrung in Neuronale Netzwerke in Python
%    Neural Networks mit Scikit
%    Naive-Bayes-Klassifikator, Grundlagen und einfache Implementierungen in Python
%    Naive-Bayes-Klassifikator mit Scikit
%    Einf�hrung in die Text-Klassifikation mit Naive-Bayes-Klassifikator
%    Python-Implementierung der Textklassifikation 

