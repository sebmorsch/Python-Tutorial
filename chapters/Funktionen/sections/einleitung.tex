In diesem Kapitel gehen wir auf die Nutzung von Funktionen in Python ein.
Eine Funktion bildet einen Code-Block ab, also eine Sequenz von Befehlen, 
welche eine bestimmte \textbf{Funktion} erf�llt. 


Dieser Code-Block wird mit dem Schl�sselwort \textit{def} gestartet,
gefolgt vom Namen der Funktion, anschlie�end von Klammern
(welche Input-Parameter beinhalten k�nnen) und zum Schluss der Funktionsdefinition
folgt noch ein Doppelpunkt. 
Nach dem Doppelpunkt kommt die Befehlssequenz. Soll die Funktion
Werte zur�ck liefern, dann steht am Ende der Sequenz das
\textit{return}-Schl�sselwort. Achtung, eine Funktion liefert immer
einen Wert zur�ck, wird keiner angegeben, so wird der Wert \textit{None}
als R�ckgabewert festgelegt.


\begin{lstlisting}[caption=Definition einer Funktion, label=funcDef]
def funktionsName (parameter):
  ...
  return rueckgabeWert
\end{lstlisting}


