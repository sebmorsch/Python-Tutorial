\begin{enumerate}
\item Kann aus der \hyperref[threads:lst:counter_condition_variable_example]{\lstinline$Counter$-Klasse}
�bernommen werden.

\item Da nicht sichergestellt werden kann, dass durch \lstinline$notify()$ genau der eine Thread 
aufgeweckt wird, dessen Bedingung als n�chstes erf�llt ist, muss \lstinline$notify_all()$ aufgerufen 
werden, nachdem der \lstinline$Counter$ inkrementiert wurde.

\item Im Konstruktor wird ein neues Attribut \lstinline$digit$ definiert dass mit dem jeweiligen Parameter
initialisiert wird. In der \lstinline$for$-Schleife, die die \lstinline$IncrementerThreads$ erzeugt, muss der 
Konstruktoraufruf entsprechend angepasst werden.

\item Die neue \lstinline$check_condition()$-Methode gibt den Wert des Ausdrucks 
\lstinline$self.counter.count \% 10 == self.digit$ zur�ck.

\item In der \lstinline$for$-Schleife innerhalb von \lstinline$run()$ wird ein \lstinline$with$-Statement 
erg�nzt. Es beinhaltet den Aufruf von \lstinline$wait_for()$, der die Methode \lstinline$check_condition()$
�bergeben bekommt, und den Aufruf von \lstinline$increment()$.

\item Eine Ausgabe in \lstinline$check_condition()$ und eine Ausgabe vor \lstinline$increment()$ gen�gen,
um das Geschehen nachzuvollziehen. Es sollten die Werte des \lstinline$Counters$ und des Attibuts
\lstinline$digit$ ausgegeben werden.
\end{enumerate}
