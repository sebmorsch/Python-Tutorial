\begin{enumerate}
\item Nach dem Erzeugen des Tuple wird eine List mit identischen Zahlwerten in identischer Reihenfolge angelegt und dieser zwei Elemente hinzugef�gt. Danach wird das Tuple �berschrieben und ausgegeben. 
\lstinputlisting[language=Python, firstline=1,lastline=6]{exercises/src/Collections/CollectionsAufgabe2Tuple.py}

\item Eine List wird mit den in Strings umgewandelten Elementen bef�llt. Mit Hilfe dieser, wird das Tuple �berschrieben und ausgegeben.
\lstinputlisting[language=Python, firstline=8,lastline=14]{exercises/src/Collections/CollectionsAufgabe2Tuple.py}
Die Ausgabe sollte wie folgt aussehen:
\begin{lstlisting}[language=Python]
('hallo', '1', '2.1', 'False', 'string')
\end{lstlisting}
Der Typ der Elemente kann auch �ber die \lstinline$type$-Methode gepr�ft werden.
\lstinputlisting[language=Python, firstline=16,lastline=17]{exercises/src/Collections/CollectionsAufgabe2Tuple.py}
\end{enumerate}