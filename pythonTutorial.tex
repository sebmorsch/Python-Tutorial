% -----------------------------------------------------------------------
% Python Tutorial
% WS 18/19
% -----------------------------------------------------------------------
\documentclass[enabledeprecatedfontcommands, fontsize=12pt,
     open=right, a4paper,
     twoside, DIV=11,
     abstractoff,
     headsepline,
     numbers=noenddot,
     BCOR=15mm,
     headings=standardclasses,
     headings=big]{scrbook}
\KOMAoptions{cleardoublepage=empty}

% Header f�r Variablen
% Variablen 
\newcommand{\theSemester}{WS18/19}
% Variable f�r die Vorlesung
\newcommand{\theProject}{Projekt Collaborative Writing}
% Variable f�r den Studiengang
\newcommand{\theTitle}{Python Tutorial}
% Variable f�r den Hochschul-Namen
\newcommand{\theSchool}{Hochschule Kaiserslautern}
% Variable f�r den Dozenten
\newcommand{\theAuthor}{
	Julian	Bernhart,
	Manfred	Brill,
	Eric Brunk,
	Mathias Fedder,
	Robin Marc Guth,
	Rainer Haffner,
	Kathrin	Hentschel,
	Fabian Kalweit,
	Kevin Konrad,
	Philipp	Lauer,
	Miriam 	Lohm�ller,
	Pascal 	Pries,
	Anatoli	Sch�fer,
	Denis Schlusche,
	Christoph Seibel
}

%
\input{setup}
%
% Schalter f�r das Ein- und Ausblenden der L�sungen
\setboolean{solutions}{true}
%
\input{pdfsetup}
%
% Beginn Dokument
%
\begin{document}
\titelseiteMitBild{\theTitle{}}
%
% Inhaltsverzeichnis
%
\tableofcontents
\clearevenpage
\pagenumbering{arabic}
%
%
%

%\uebungZwei{beispielAufgabe1}{grundlagen01}{grundlagen02}

% !TeX root = ../pythonTutorial.tex
\chapter{Grundlagen}

% !TeX root = ../pythonTutorial.tex

\section{Elementare Datentypen}

�hnlich wie zu Java und C oder C++, gibt es auch in Python Variablen. Allerdings gibt es dabei immense Unterschiede zu den anderen Programmiersprachen, weshalb sich ein genauerer Blick auf die einzelnen Datentypen in jedem Fall lohnt.
In den uns bekannten Sprachen muss eine Variable einem bestimmten Datentyp zugeordnet (deklariert) werden. Der Datentyp kann darauf folgend zur Laufzeit nicht wieder ge�ndert werden, der Wert innerhalb des Datentyps allerdings schon. Ist eine Variable einmal zum Beispiel dem Datentyp Integer zugeordnet, so l�sst sich der Variable anschlie�end nicht mehr in einen String-Wert umwandeln. In Python hingegen, ist dies ohne weiteres m�glich. Hier wird n�mlich g�nzlich auf eine explizite Typdeklaration verzichtet. Zeigt eine Variable beispielsweise auf eine ganze Zahl, so wird diese als ein Objekt vom Typ Integer interpretiert. Allerdings kann man sie im n�chsten Schritt einfach auf ein String-Objekt zeigen lassen. 
Soviel zum allgemeinen Unterschied zu den anderen Programmiersprachen. Betrachten wir nun die Datentypen etwas genauer.


\input{./chapters/functionsAndModules.tex}

% !TeX root = ../pythonTutorial.tex
\chapter{Benutzeroberfl�chen}

% !TeX root = ../pythonTutorial.tex
\chapter{Python Bibliotheken}

\label{bibliotheken:numpy}
\section{NumPy}

NumPy ist eine Python-Bibliothek f�r wissenschaftliches Rechnen.
Sie beinhaltet unter anderem Folgendes:

\begin{itemize}
  \item m�chtige $n$-dimensionale Array-Objekte
  \item Werkzeuge zur Integration von C und Fortran
  \item Lineare Algebra, Fouriertransformation, Erzeugung von Zufallszahlen
\end{itemize}

% Zitat: http://www.numpy.org

Um NumPy zu installieren, kann der Befehl \lstinline!pip install numpy!
verwendet werden.

\subsection{Arrays}

Der Array-Datentyp von NumPy hei�t \lstinline!numpy.ndarray!. Anders als Pythons
Listentyp \lstinline!list! unterst�tzt \lstinline!numpy.ndarray! numerische
Operationen auf Arrays. Die Grundrechenarten werden zwischen zwei Arrays
elementweise und zwischen Array und \lstinline!int!/\lstinline!float! f�r alle
Elemente des Arrays durchgef�hrt.

So kann etwa jeder Wert in einem Array mit den folgenden Anweisungen um drei
erh�ht werden:
\begin{lstlisting}
import numpy as np
a = np.array([1,2,3])
a + 3 # [4 5 6]
\end{lstlisting}

Subtraktion, Multiplikation, Division, Ganzzahldivision und Potenzieren
funktionieren analog:\footnote{Der Import von \lstinline!numpy! wird der
�bersichtlichkeit halber nachfolgend ausgelassen.}
\begin{lstlisting}
a = np.array([1,2,3])
a - 3  # [-2 -1  0]
a * 3  # [3 6 9]
a / 3  # [0.33333333 0.66666667 1.        ]
a // 3 # [0 0 1]
a ** 3 # [ 1  8 27]
\end{lstlisting}

Zwei Arrays gleicher L�nge k�nnen elementweise miteinander verkn�pft werden:
\begin{lstlisting}
a = np.array([1,2,3])
b = np.array([4,5,6])
a + b  # [5 7 9]
a - b  # [-3 -3 -3]
a * b  # [ 4 10 18]
a / b  # [0.25 0.4  0.5 ]
a ** b # [  1  32 729]
a // b # [0 0 0]
\end{lstlisting}

Um ein NumPy-Array zu erzeugen, wird ein \lstinline!list!-Objekt an
\lstinline!np.array()! �bergeben, dabei wird der in der \lstinline{list}
enthaltene Datentyp in einem Datentyp von NumPy konvertiert. Um den Datentyp
eines Arrays herauszufinden, wird \lstinline!.dtype.name! genutzt. Anders
als bei \lstinline!list! m�ssen s�mtliche Elemente eines Arrays den gleichen
Typ haben.
\begin{lstlisting}
a = np.array([1,2,3])
a.dtype.name # 'int64'
b = np.array([1.4,2.5,3.6])
a.dtype.name # 'float64'
\end{lstlisting}

Wenn \lstinline!int! und \lstinline!float! gemischt �bergeben werden,
konvertiert NumPy in den Flie�kommatyp. Wie viele Bit f�r einen Integer bzw. ein
Float zur Verf�gung stehen, ist von der Prozessorarchitektur abh�ngig. Moderne
Computer unterst�zten in der Regel eine Gr��e von 64 Bit.


\subsection{Konstanten und Funktionen}
Es stehen f�r die mathematische Anwendungen auch Konstanten zur Verf�gung,
darunter die Folgenden mit den entsprechenden Werten und Pr�zisionen mit
\lstinline!float64!:
\begin{lstlisting}
>>> np.pi
3.141592653589793
>>> np.e
2.718281828459045
>>> np.euler_gamma
0.5772156649015329
>>> np.PINF
inf
>>> np.NINF
-inf
>>> np.NAN
nan
>>> np.PZERO
0.0
>>> np.NZERO
-0.0
>>> np.NAN
nan
\end{lstlisting}

\lstinline!np.NZERO! steht f�r die negative Darstellung der Null bei
Flie�kommazahlen, \lstinline!np.PZERO! f�r die positive Darstellung.

NumPy unterst�tzt eine Vielzahl an mathematischen Funktionen, darunter unter
anderem trigonometrische Funktionen, Rundungs-,
Summations- und Multiplikationsfunktionen und Funktionen zur Behandlung
komplexer Zahlen.

Die grundlegenden trigonometrischen Funktionen sind selbsterkl�rend, sie werden
elementweise auf das Array angewendet:
\begin{lstlisting}
>>> a = np.array([0, np.pi/6, np.pi/4, np.pi/3, np.pi/2])
>>> np.sin(a)
[0.         0.5        0.70710678 0.8660254  1.        ]
>>> np.cos(a)
[1.00000000e+00 8.66025404e-01 7.07106781e-01 5.00000000e-01
6.12323400e-17]
>> np.tan(a)
[0.00000000e+00, 5.77350269e-01, 1.00000000e+00,
1.73205081e+00, 1.63312394e+16]
\end{lstlisting}

Auch die Umrechung von Radians in Grad
\begin{lstlisting}
>>> a = np.array([0, np.pi/6, np.pi/4, np.pi/3, np.pi/2])
>>> np.degrees(a)
[ 0., 30., 45., 60., 90.]
\end{lstlisting}
und Grad in Radians m�glich:
\begin{lstlisting}
>>> a = np.array([ 0, 30, 45, 60, 90])
>>> np.radians(a)
[0.         0.52359878 0.78539816 1.04719755 1.57079633]
\end{lstlisting}

Mit \lstinline!np.around! k�nnen s�mtliche Werte im Array auf eine bestimmte
Anzahl von Stellen gerundet werden. Ohne Angabe eines zweiten Arguments wird
kaufm�nnisch auf die n�chste Ganzzahl gerundet.
\begin{lstlisting}
>>> a = np.array([1.49, 1.5, 1.51])
>>> np.round(a)
[1. 2. 2.]
\end{lstlisting}
Mit dem optionalen zweiten Argument wird die Anzahl an Nachkommastellen, auf die
gerundet werden soll, angegeben:
\begin{lstlisting}
>>> a = np.array([1.25, 1.53, 1.99])
>>> np.round(a, 1)
[1.2, 1.5, 2. ]
\end{lstlisting}

Um alle Elemente eines Arrays aufzusummieren, wird die Funktion \\
\lstinline!np.sum()! verwendet.
\begin{lstlisting}
>>> a = np.array([1, 2, 3])
>>> np.sum(a)
6
\end{lstlisting}
Mit \lstinline!np.prod()! k�nnen die Elemente der Liste miteinander
multipliziert werden.
\begin{lstlisting}
>>> a = np.array([2, 3, 4])
>>> np.prod(a)
24
\end{lstlisting}


Sollten \lstinline!nan! (not a number) im Array vorkommen k�nnen, so kann
\lstinline!np.nansum! beziehungsweise \lstinline!np.nanprod! verwendet werden.
Bei \lstinline!np.nansum! werden \lstinline!nan! als \lstinline!0!
interpretiert,
\begin{lstlisting}
>>> a = np.array([np.NAN, 1, 2, 3])
>>> np.sum(a)
nan
>>> np.nansum(a)
6.0
\end{lstlisting}
bei \lstinline!np.nanprod! als \lstinline!1!:
\begin{lstlisting}
>>> a = np.array([np.NAN, 2, 3, 4])
>>> np.prod(a)
nan
>>> np.nanprod(a)
24.0
\end{lstlisting}
Das Ergebnis der Addition bzw. Multiplikation ist vom Typ \lstinline!float64!.
\lstinline!nan! ist ein valider Flie�zahlwert, daher werden die restlichen Werte
in der zu konvertierenden \lstinline!list! von \lstinline!int! zu
\lstinline!float64! umgewandelt.



% !TeX root = ../pythonTutorial.tex
\chapter{Weiterf�hrende Themen}
\section{Maschinelles Lernen in Python}
Das Themengebiet des maschinellen Lernens kann verschiedene Stufen von Komplexit�t erreichen. Grundlegend ist das mathematische Verst�ndnis �ber die verschiedenen im Machine Learning eingesetzten Algorithmen. Diese werden in diesem Tutorial nicht beschrieben.\\
Ist man mit diesen vertraut und m�chte diese nun mittels Python anwenden, bietet es sich zum Einstieg an mit kleinen Projekten zu starten. Hierbei sollte sich bereits zu Beginn eine Vorgehensweise �berlegt werden, wie auch bei allen anderen Projekten. Dies kann den Grundstein daf�r legen, was �berhaupt ben�tigt wird. Python bietet im Bereich des maschinellen Lernens viele unterschiedliche M�glichkeiten an, sodass man sich fr�hzeitig Gedanken �ber den Aufbau machen sollte. Grob kann ein Projekt in 5 Schritt aufgeteilt werden, anhand derer man sp�ter das Ergebnis verifizieren kann.\\

1. Problem definieren\\
2. Daten vorbereiten\\
3. Algorithmen evaluieren\\
4. Ergebnisse verbessern\\
5. Ergebnisse darstellen\\


Eine weit verbreitete M�glichkeit ist das Einbinden von bereits existierenden Bibliotheken, die bereits gewissen Funktionalit�ten von Haus auf anbieten. Eine eigene Nachbildung von verbreiteten Algorithmen aus dem Bereich Machine Learing ist daher meist nicht n�tig.\\
Eine zweite M�glichkeit ist die Integration von R. Bei R handelt es sich um eine eigene Programmiersprache, welche den Schwerpunkt in mathematischen Probleml�sungen hat. Python und R lassen sich beide sowohl eigenst�ndig, also auch in Verbindung miteinander einsetzen.

Im Folgenden Abschnitt gibt es eine �bersicht, �ber wichtige und bekannte Bibliotheken aus dem Bereich maschinelles Lernen. Die Anzahl der Bibliotheken macht ein Einstieg nicht ganz leicht, sodass man sich die St�rken und Schw�chen der einzelnen Bibliotheken betrachten sollte. 


\subsection{Bekannte Bibliotheken}
Im Bereich maschinelles Lernen sind schon viele Bibliotheken vorhanden, die unterschiedliche Schwerpunkte in dem Bereich bedienen. Aus diesem Grund erfolgt zuerst eine �bersicht �ber verbreitete Bibliotheken.


\textbf{Download und Installation}\\
Alle Bibliotheken die genutzt werden sollen m�ssen zuerst installiert werden. Hierf�r gibt es je nach Bibliothek und teilweise je nach Betriebssystem mehrere Wege. Es wird empfohlen hier aus der jeweiligen Webseite die geeignetsten Variante zu w�hlen und auszuf�hren.

\textbf{Versionen und Kompatibilit�t}\\
Nach der Installation sollten alle Versionen ausgelesen und abgeglichen werden. Die Kompatibilit�t ist nicht durchgehend gew�hrleistet, das betrifft vor allem die unterschiedlichen Python-Versionen.

Der folgende Codeausschnitt zeigt ein Beispiel. Dieser kann entweder direkt in der Eingabeaufforderung bzw. Console nach Start von Python ausgef�hrt werden oder innerhalb einer IDE.

\begin{lstlisting}
# Python version
import sys
print("Pyton: {}".format(sys.version))

# numpy version
import numpy
print("numpy: {}".format(numpy.__version__))
\end{lstlisting}

Die Ausgabe ist beispielsweise die folgende:
\begin{lstlisting}
Pyton: 3.6.1 (v3.6.1:69c0db5, Mar 21 2017, 17:54:52) 
	[MSC v.1900 32 bit (Intel)]

numpy: 1.13.3 
\end{lstlisting}

Auf diese Art und Weise sollte alle Bibliotheken gepr�ft werden, da so eventuelle Kompabilit�tsprobleme oder fehlerhafte Installation erkannt werden k�nnen.

\subsubsection{Bibliotheken}
Grob kann man zwischen Datenanalyse und Visualisierung unterscheiden. Zwei bekannte Bibliotheken aus dem Bereich Datenanalyse sind \textit{numpy} und \textit{pandas}, mit denen beispielsweise Gleichungen und Optimierungsprobleme gel�st werden k�nnen. Au�erdem lassen sich Integrale berechnen, statistische Berechnungen durchf�hren und auch simulieren. All das wird f�r maschinelles Lernen ben�tigt. Da die Berechnungen mit Routinen nah an der Hardware durchgef�hrt werden, lassen sich bei entsprechender Programmierung effiziente Programme schreiben. Ergebnisse solcher Berechnungen k�nnen mithilfe des Moduls \textit{matplotlib} visualisiert werden. ~\cite{Python3}


\textbf{numpy}\\
In diesem Modul wird ein flexibler Datentyp f�r mehrdimensionale Arrays zur Verf�gung gestellt. Dies erm�glicht eine effiziente Durchf�hrung von komplexen Rechnungen.
~\cite{numpyreference}

Die Arrays in \textit{numpy} sind dreidimensional und k�nnen so eine Vielzahl von Anwendungsf�llen abbilden. Der Fokus von \textit{numpy} liegt in der Datenhaltung und Manipulation von Daten. Hier speziell die numerische Manipulation aus dem Bereich der linearen Algebra. Mit den Matrizen k�nnen beispielsweise Multiplikationen und Dekompositionen durchgef�hrt werden. Aus diesem Grund sind \textit{numpy}-Array oft die Datenstruktur, mit der weiterf�hrende Bibliotheken arbeiten k�nnen.


\textbf{pandas}\\
Die Webseite zum Paket ~\cite{pandas} beschreibt dieses Paket als gute Wahl zur schnellen und flexiblen Aufbereitung von Daten. Pandas bietet hier M�glichkeiten, um schnell auf Eintr�ge zuzugreifen. Dies ist m�glich, da mittels Pandas Serien und Dataframes erzeugt werden k�nnen, die im Gegensatz zu einem Array in Python auch Spaltentitel und Indexes k�nnen.

Die describe()-Methode bietet hierbei einen ersten �berblick �ber die im Dataframe enthalten Daten. Ohne weitere Programmierung werden Informationen wie Maximalwert, Minimalwert und Durchschnitte f�r jede Spalte berechnet und angezeigt. 

Spalten und Zeilen k�nnen mittels Pandas gefiltert, erweitert und ver�ndert werden, sodass Pandas oft im ersten Schritt genutzt wird, um die auszuwertenden Daten zu laden und genauer analysieren zu k�nnen.

\textbf{scipy}\\
Erg�nzend bzw. aufbauen auf \textit{numpy} werden durch scipy viele mathematische Operationen bereit gestellt. 
Das Modul \textit{scipy} ist sehr m�chtig und daher nochmal in Untermodule aufgeteilt. Innerhalb der Untermodule werden bestimmte Funktionalit�ten gruppiert. Die genauen Untermodule k�nnen aus der Onlinedokumentation ~\cite{scipyreference}


\textbf{scikit-learn}\\
Viele DataMining bzw. Machine Learning Funktionalit�ten werden bereits durch die \textit{scikit-learn} Bibliothek (manchmal aus sklearn abgek�rzt) zur Verf�gung gestellt. Die klassischen Algorithmen, wie \textit{k-Means} oder \textit{knn} sind bereits integriert. Des Weiteren ist auch mit \textit{scikit-learn} eine Aufbereitung der Daten m�glich. Hier werden beispielsweise Normalierung und Skalierung unterst�tzt. Insgesamt handelt es sich um eine sehr m�chtige Bibliothek, mit der es m�glich ist unterschiedliche Algorithmen zu probieren und verschiedene Test- und Trainingsverfahren zu zu testen. Es k�nnen auch neue Daten anhand der gelernten Modelle klassifiziert und vorhergesagt werden.~\cite{scikit}


\textbf{rpy2}\\
Eine weitere M�glichkeit ist das Einbinden des Pakets \textit{rpy2}. Hierbei handelt es sich um eine Bibliothek, welche Komponenten aus R zur Verf�gung stellt.~\cite{rpy2}


\textbf{matplotlib}\\
Mit dem Modul \textit{matplotlib} k�nnen Daten in einem Diagramm dargestellt werden. Hiermit kann ein erstes Verst�ndnis der Daten oder Ergebnisse erreicht werden. Es werden unter anderem Liniendiagramme, Histogramme, Balkendiagramme aber auch Heatmaps unterst�tzt. Hier k�nnen sowohl Achsen, Farben und auch Beschriftungen nach Bedarf angepasst werden. \textit{matplotlib} unterst�tzt sowohl \textit{pandas} als auch \textit{numpy} und ist oft daher bereits am Anfang von hoher Bedeutung, um einen �berblick �ber die Daten zu bekommen. ~\cite{matplotlib}


%
% Literatur
%
\cleardoublepage
\phantomsection
\addcontentsline{toc}{chapter}{Literaturverzeichnis}
\chaptermark{Literaturverzeichnis}
\sectionmark{Literatur}\label{literatur}
\bibliography{./bib/literatur}
%
% Anhang
%
\appendix

\chapter{L�sungshinweise}\label{solhinweise}
Hier finden sich die L�sungshinweise zu den Aufgaben.

\section{L�sungen zu Grundlagen}
\hinweis{grundlagen01}
\hinweis{grundlagen02}

\section{L�sungen zu Datentypen und Kontrollstrukturen}
\hinweis{DatatypesAufgabe1}
\hinweis{ifelseAufgabe1}
\hinweis{KontrollstrukturenAufgabe1}


\section{L�sungen zu Collections}
\hinweis{Collections/CollectionsAufgabe1List}
\hinweis{Collections/CollectionsAufgabe2List}
\hinweis{Collections/CollectionsAufgabe1Tuple}
\hinweis{Collections/CollectionsAufgabe2Tuple}
\hinweis{Collections/CollectionsAufgabe1Set}
\hinweis{Collections/CollectionsAufgabe1Dictionary}

\hinweis{beispielAufgabe1}

\section{L�sungen zu Funktionen und Module}
\input{./exercises/solutions/functionsAndModules/exercisesFunctionsAndModules.tex}

\section{L�sungen zu Maschinelles Lernen}
\hinweis{machinelearning1}
\hinweis{machinelearning2}
%
% Index
%\clearevenpage
%\phantomsection
%\small
%\chaptermark{Index}
%\sectionmark{Index}
%\printindex
%\normalsize
\end{document}
